% phpMyAdmin LaTeX Dump
% version 4.9.1
% https://www.phpmyadmin.net/
%
% ホスト: localhost
% 生成日時: 2020 年 6 月 21 日 18:19
% サーバのバージョン: 5.7.27-log
% PHP のバージョン: 7.2.24
% 
% データベース: 'xcj1p_lab'
% 

%
% 構造: wp_download_monitor_files
%
 \begin{longtable}{|l|c|c|c|l|} 
 \caption{テーブル wp\_download\_monitor\_files の構造} \label{tab:wp_download_monitor_files-structure} \\
 \hline \multicolumn{1}{|c|}{\textbf{カラム}} & \multicolumn{1}{|c|}{\textbf{データ型}} & \multicolumn{1}{|c|}{\textbf{NULL}} & \multicolumn{1}{|c|}{\textbf{デフォルト値}} & \multicolumn{1}{|c|}{\textbf{コメント}} \\ \hline \hline
\endfirsthead
 \caption{テーブル wp\_download\_monitor\_files の構造 (続き)} \\ 
 \hline \multicolumn{1}{|c|}{\textbf{カラム}} & \multicolumn{1}{|c|}{\textbf{データ型}} & \multicolumn{1}{|c|}{\textbf{NULL}} & \multicolumn{1}{|c|}{\textbf{デフォルト値}} & \multicolumn{1}{|c|}{\textbf{コメント}} \\ \hline \hline \endhead \endfoot 
\textbf{\textit{id}} & int(10) & いいえ &  \\ \hline 
title & varchar(200) & いいえ &  \\ \hline 
filename & longtext & いいえ &  \\ \hline 
file\_description & longtext & はい & NULL \\ \hline 
dlversion & varchar(200) & いいえ &  \\ \hline 
postDate & datetime & いいえ &  \\ \hline 
hits & int(12) & いいえ &  \\ \hline 
user & varchar(200) & いいえ &  \\ \hline 
members & int(1) & はい & NULL \\ \hline 
mirrors & longtext & はい & NULL \\ \hline 
 \end{longtable}

%
% データ: wp_download_monitor_files
%
 \begin{longtable}{|l|l|l|l|l|l|l|l|l|l|} 
 \hline \endhead \hline \endfoot \hline 
 \caption{テーブル wp\_download\_monitor\_files の内容} \label{tab:wp_download_monitor_files-data} \\\hline \multicolumn{1}{|c|}{\textbf{id}} & \multicolumn{1}{|c|}{\textbf{title}} & \multicolumn{1}{|c|}{\textbf{filename}} & \multicolumn{1}{|c|}{\textbf{file\_description}} & \multicolumn{1}{|c|}{\textbf{dlversion}} & \multicolumn{1}{|c|}{\textbf{postDate}} & \multicolumn{1}{|c|}{\textbf{hits}} & \multicolumn{1}{|c|}{\textbf{user}} & \multicolumn{1}{|c|}{\textbf{members}} & \multicolumn{1}{|c|}{\textbf{mirrors}} \\ \hline \hline  \endfirsthead 
\caption{テーブル wp\_download\_monitor\_files の内容 (続き)} \\ \hline \multicolumn{1}{|c|}{\textbf{id}} & \multicolumn{1}{|c|}{\textbf{title}} & \multicolumn{1}{|c|}{\textbf{filename}} & \multicolumn{1}{|c|}{\textbf{file\_description}} & \multicolumn{1}{|c|}{\textbf{dlversion}} & \multicolumn{1}{|c|}{\textbf{postDate}} & \multicolumn{1}{|c|}{\textbf{hits}} & \multicolumn{1}{|c|}{\textbf{user}} & \multicolumn{1}{|c|}{\textbf{members}} & \multicolumn{1}{|c|}{\textbf{mirrors}} \\ \hline \hline \endhead \endfoot
1 & 【3Dディスプレイの新方式「2x3D」の開発に成功:2Dと3Dを同スクリーンで上映】 & http://blog.shirai.la/wp-content/uploads/downloads/2012/11/PressRelease2x3D-20121025.pdf & 2012年10月25日(木)
報道関係各位・プレスリリース:神奈川工科大学 情報メディア学科 白井研究室

【3Dディスプレイの新方式「2x3D」の開発に成功:2Dと3Dを同スクリーンで上映】 & 2 & 2012-11-01 10:10:00 & 511 & admin & 0 &  \\ \hline 
2 & 2x3D: Real-Time Shader for Simultaneous 2D/3D Hybrid Theater & http://blog.shirai.la/wp-content/uploads/downloads/2012/11/2x3D-Abst0923SR3F.pdf & Wataru FUJIMURA, Yukua KOIDE, Robert SONGER, Takahiro HAYAKAWA, Akihiko SHIRAI, Kazuhisa YANAKA, "2x3D: Real-Time Shader for Simultaneous 2D/3D Hybrid Theater", ACM SIGGRAPH ASIA 2012 Emerging Technologies, Singapore, 2012. &  & 2012-11-01 01:31:00 & 1640 & admin & 0 &  \\ \hline 
3 & 2X3D:2D+3D同時上映可能なハイブリッドシアター & http://blog.shirai.la/wp-content/uploads/downloads/2012/11/2x3D-vrsj2012.pdf & 藤村航,小出雄空明,早川貴奉,谷中一寿,白井暁彦, "2X3D:2D+3D同時上映可能なハイブリッドシアター", 第17回日本バーチャルリアリティ学会大会論文集, 2012年9月 &  & 2012-11-01 01:36:00 & 600 & iwadate & 0 &  \\ \hline 
4 & 日経産業新聞2012年10月30日「2次元と3次元同一画面で表示」 & http://blog.shirai.la/wp-content/uploads/downloads/2012/11/39d6801b25e3fc313c5cc40cba4bccce.pdf & 日経産業新聞2012年10月30日「2次元と3次元同一画面で表示」 &  & 2012-11-01 13:03:03 & 1 & shirai & 1 &  \\ \hline 
5 & 卒論TeXテンプレート & http://blog.shirai.la/wp-content/uploads/downloads/2012/11/SotsuronTeX.zip & とりあえず作ってみたレベル,でも亀山くんには十分使えるのではないでしょうか.修士にもおすすめ. & 1 & 2012-11-01 20:58:05 & 0 & admin & 1 &  \\ \hline 
6 & (非公開)ScritterHGX converter & http://blog.shirai.la/wp-content/uploads/downloads/2012/11/Scritter3D.zip & for Processing 2.x

ソースの場所はこちら
DropboxLabo2012NEC-DS

ソースに書かれているPNG画像を起動時に読み込んで(30秒ぐらい待つこと)、a-s-w-dでアフィン変換、1-8が頂点の指定です。
Cキーでモード切替(方眼・方眼・3D・HGX)、Pキーで画像書き出しです。
\{left,right,shot\}.pngのうちleft.pngとright.pngをつないだものが明日使う画像です。

これから画像吐き出し部分の修正に入りますが、他の仕事もあるので余りはかどらないかも…基本的には
C:UsersakiDropboxLabo2012NEC-DSScritterGXgen
にあるソースの合体です。 & v.1.00 & 2012-11-08 00:11:40 & 0 & admin & 1 &  \\ \hline 
7 & KinectTVRemote+AccuMotionLib & http://blog.shirai.la/wp-content/uploads/downloads/2012/11/KinectTVRemote1.zip & CQ出版社「インターフェース」2013年1月号「研究 人間センシング」の特集関連記事『誤認識しにくいジェスチャ入力の研究!』のためのサンプルプログラムです。

本誌記事の詳細はこちらをお読みください。
http://www.kumikomi.net/interface/contents/201301.php
http://www.kumikomi.net/interface/sample/201301/if01\_130.pdf & 1.00 & 2012-11-22 16:24:00 & 716 & admin & 0 &  \\ \hline 
8 & フランスからのアート系研究者が特別講演(2012/12/3 5限)  & http://blog.shirai.la/wp-content/uploads/downloads/2012/11/FrenchGameArtistPoster20121119.pdf & 2012年度 特別講演企画
「ビデオゲームおよびVR環境における感情表現」
講師:ゲームアーティスト:Erik GESLIN(エリック・ジェラン、FRANCE)

【概要】
アメリカ・ハリウッドに並び、CG・アニメーション制作の先進国であるフランスから、本学の提携校であるENSAM所属のアート系研究者Erik GESLIN氏による講演会で、内容は世界最高峰のCG・インタラクティブ技術の国際会議「SIGGRAPH ASIA」での最新の講演内容「ビデオゲームおよびVR環境における感情誘発の基本手法と複雑化」および、フランスにおけるゲーム・VR産業、CGクリエイター教育、就職環境についてのお話をいただきます。豊富なビジュアル、英語、フランス語による講演に加え、日本語同時通訳での講演の予定。幅広い学生の参加を期待します。

【対象】   情報メディア学科学部生・院生全学年
【日時場所】 12月3日 (月) 5限(16:40~18:10) K-1棟2F 202教室

【講演・講師概要】
感情の哲学は古くから、プラトン、孔子、デカルトとスピノザによって扱われ、19世紀にダーウィンとジェームズにより科学的研究となり、現代の科学技術では脳科学において研究されています。この基本的な知識使い、古代と現代美術の歴史の中で、感情を誘導するアート手法を研究しています.ゲームプレイやビデオゲームのスクリプトの実装を通して、恐怖や喜びだけでなく、共感や同情といった複雑で基本的な感情を誘発するメソッドを紹介します。講師Erik Geslin はMicroïd(英国)、 Ubisoft(フランス)、Leon Brothers(フランス)、VRアプリ開発社「Nautilus」、日本の「Raingraph」におけるアートデザイン、ディレクションを担当、10年以上にわたる経験をもつ.教育者として国立工芸大学(ENSAM Paris Tech、本学提携校)、およびデジタルインタラクティブクリエイション高専(ESCIN)における感情表現系アートの講師、指導学生が国際VR作品コンテストIVRCにおいて2006年、 2012年にフランス代表に選出されています。

■本件に関するお問い合せは白井まで &  & 2012-11-26 11:06:41 & 595 & shirai & 0 &  \\ \hline 
9 & IVRC2010学内説明会ポスター & http://blog.shirai.la/wp-content/uploads/downloads/2012/11/IVRCpre201004-KAIT.pdf & IVRC2010IVRC2010学内説明会のポスター &  & 2012-11-28 03:28:28 & 504 & kitada & 0 &  \\ \hline 
10 & 2x3D-Abst0923SR3F.pdf & http://blog.shirai.la/wp-content/uploads/downloads/2012/11/2x3D-Abst0923SR3F.pdf & Wataru FUJIMURA, Yukua KOIDE, Robert SONGER, Takahiro HAYAKAWA, Akihiko SHIRAI, Kazuhisa YANAKA, "2x3D: Real-Time Shader for Simultaneous 2D/3D Hybrid Theater", ACM SIGGRAPH ASIA 2012 Emerging Technologies (accepted), Singapore, 2012. &  & 2012-11-28 05:25:01 & 11 & kitada & 0 &  \\ \hline 
11 & エンタテイメントシステムにおける加速度センサを用いたユーザ解析と非言語評価手法の提案 & http://blog.shirai.la/wp-content/uploads/downloads/2012/11/H23-Takumi.pdf & エンタテイメントシステムにおける加速度センサを用いたユーザ解析と非言語評価手法の提案 &  & 2012-11-28 06:22:28 & 916 & kitada & 0 &  \\ \hline 
12 & 次世代エンタテイメントシステム開発のためのプロトタイピング手法 & http://blog.shirai.la/wp-content/uploads/downloads/2012/11/H23-Yamashita.pdf & 次世代エンタテイメントシステム開発のためのプロトタイピング手法 &  & 2012-11-28 06:27:36 & 1479 & kitada & 0 &  \\ \hline 
13 & 骨格情報を用いたNUIにおける認識アルゴリズムの開発 & http://blog.shirai.la/wp-content/uploads/downloads/2012/11/H23-Fujimura.pdf & 骨格情報を用いたNUIにおける認識アルゴリズムの開発 &  & 2012-11-29 04:33:13 & 2777 & kitada & 0 &  \\ \hline 
14 & 霧箱を使った科学コミュニケーションのための放射線可視化システムの開発 & http://blog.shirai.la/wp-content/uploads/downloads/2012/11/H23-Kitada.pdf & 霧箱を使った科学コミュニケーションのための放射線可視化システムの開発 &  & 2012-11-29 04:37:46 & 889 & kitada & 0 &  \\ \hline 
15 & 脈波の物理的可視化によるノンバーバル・コミュニケーション手法の開発 & http://blog.shirai.la/wp-content/uploads/downloads/2012/11/H23Y-Miyakawa.pdf & 脈波の物理的可視化によるノンバーバル・コミュニケーション手法の開発 &  & 2012-11-29 04:41:01 & 945 & kitada & 0 &  \\ \hline 
16 & ペルソナ法を用いたブログ形式による商品案内サイト構築 & http://blog.shirai.la/wp-content/uploads/downloads/2012/11/H23Y-Tsuzuku.pdf & ペルソナ法を用いたブログ形式による商品案内サイト構築 &  & 2012-11-29 04:44:42 & 743 & kitada & 0 &  \\ \hline 
17 & 映像多重化システムと速度変化に注目したe-sports観戦での高い一体感を生み出す演出手法の提案と開発 & http://blog.shirai.la/wp-content/uploads/downloads/2012/11/H22-Arahara.pdf & 映像多重化システムと速度変化に注目したe-sports観戦での高い一体感を生み出す演出手法の提案と開発 &  & 2012-11-29 04:47:48 & 742 & kitada & 0 &  \\ \hline 
18 & 測域センサを用いたエンタテイメントシステムにおける遊戯状態の可視化による人の自然な振る舞いの物理評価(特別改訂版) & http://blog.shirai.la/wp-content/uploads/downloads/2012/11/H22-iwadate.pdf & 測域センサを用いたエンタテイメントシステムにおける遊戯状態の可視化による人の自然な振る舞いの物理評価(特別改訂版) &  & 2012-11-29 04:50:12 & 803 & kitada & 0 &  \\ \hline 
19 & デジタルシネマにおけるHDR画像の性質を使ったデバイス非依存のカラーマネジメント手法 & http://blog.shirai.la/wp-content/uploads/downloads/2012/11/H22-syu.pdf & デジタルシネマにおけるHDR画像の性質を使ったデバイス非依存のカラーマネジメント手法 &  & 2012-11-29 04:53:50 & 759 & kitada & 0 &  \\ \hline 
20 & UbiCode:パブリックディスプレイへのバーチャルインタラクティビティの追加 & http://blog.shirai.la/wp-content/uploads/downloads/2012/11/VRSJ2012-UbiCode.pdf & UbiCode:パブリックディスプレイへのバーチャルインタラクティビティの追加 &  & 2012-11-29 05:07:00 & 880 & admin & 0 &  \\ \hline 
21 & 2X3D:2D+3D同時上映可能なハイブリッドシアター & http://blog.shirai.la/wp-content/uploads/downloads/2012/11/2x3D-vrsj20121.pdf & 2X3D:2D+3D同時上映可能なハイブリッドシアター &  & 2012-11-29 05:13:00 & 861 & iwadate & 0 &  \\ \hline 
22 & RFIDとプロジェクションマッピングを活用した科学館向けエンタテイメントVRシステム & http://blog.shirai.la/wp-content/uploads/downloads/2012/11/Science-quest2012kitada.pdf & RFIDとプロジェクションマッピングを活用した科学館向けエンタテイメントVRシステム", エンタテイメントコンピューティング2012 &  & 2012-11-29 05:15:00 & 1290 & iwadate & 0 &  \\ \hline 
23 & カラオケに並列するダンスゲームのゲームデザインの提案 & http://blog.shirai.la/wp-content/uploads/downloads/2012/11/EC2012-Nara.pdf & カラオケに並列するダンスゲームのゲームデザインの提案 &  & 2012-11-29 05:22:00 & 783 & admin & 0 &  \\ \hline 
24 & 多重化隠蔽技術UbiCodeを使ったデジタルサイネージのインタラクティブ化によるコミュニケーション支援 & http://blog.shirai.la/wp-content/uploads/downloads/2012/11/HCG2012-1210.pdf & 多重化隠蔽技術UbiCodeを使ったデジタルサイネージのインタラクティブ化によるコミュニケーション支援 &  & 2012-11-29 05:42:00 & 945 & admin & 0 &  \\ \hline 
25 & PARAOKE alpha: a new application development of multiplex-hidden display technique for music entertainment system & http://blog.shirai.la/wp-content/uploads/downloads/2012/11/NICOINT2012-PARAOKE-0517.pdf & PARAOKE alpha: a new application development of multiplex-hidden display technique for music entertainment system &  & 2012-11-29 05:50:00 & 819 & kitada & 0 &  \\ \hline 
26 & AccuMotion: intuitive recognition algorithm for new interactions and experiences for the post-PC era & http://blog.shirai.la/wp-content/uploads/downloads/2012/11/VRIC2012-AccuMotion0426.pdf & AccuMotion: intuitive recognition algorithm for new interactions and experiences for the post-PC era &  & 2012-11-29 06:07:00 & 1333 & shirai & 0 &  \\ \hline 
27 & 多重化・隠蔽画像を用いたデジタルサイネージの開発 & http://blog.shirai.la/wp-content/uploads/downloads/2012/11/ITSympo2011-Koide-0215YK3.pdf & 多重化・隠蔽画像を用いたデジタルサイネージの開発 &  & 2012-11-29 06:15:00 & 1261 & iwadate & 0 &  \\ \hline 
28 & 霧箱を使った科学コミュニケーションによる放射線理解の調査手法 & http://blog.shirai.la/wp-content/uploads/downloads/2012/11/ITSympo2011-kitada.pdf & 霧箱を使った科学コミュニケーションによる放射線理解の調査手法 &  & 2012-11-29 06:23:00 & 911 & iwadate & 0 &  \\ \hline 
29 & 多重化・隠蔽サイネージを用いた次世代カラオケ・エンタテイメントシステムの提案 & http://blog.shirai.la/wp-content/uploads/downloads/2012/11/ArtSciForum2012Koide.pdf & 多重化・隠蔽サイネージを用いた次世代カラオケ・エンタテイメントシステムの提案 &  & 2012-11-29 06:30:00 & 1150 & iwadate & 0 &  \\ \hline 
30 & 霧箱の動画像処理による空間放射線可視化システム & http://blog.shirai.la/wp-content/uploads/downloads/2012/11/ArtSciForum2012kitada.pdf & 霧箱の動画像処理による空間放射線可視化システム &  & 2012-11-29 06:36:00 & 779 & iwadate & 0 &  \\ \hline 
31 & CartooNect: Sensory motor playing system using cartoon actions & http://blog.shirai.la/wp-content/uploads/downloads/2012/11/CartoonectVRIC2011F.pdf & CartooNect: Sensory motor playing system using cartoon actions &  & 2012-11-29 06:50:00 & 888 & kitada & 0 &  \\ \hline 
32 & 加速度センサを用いたエンタテインメントシステムの非言語評価手法の提案 & http://blog.shirai.la/wp-content/uploads/downloads/2012/11/EC2011\_Takumi.pdf & 加速度センサを用いたエンタテインメントシステムの非言語評価手法の提案 &  & 2012-11-29 07:03:00 & 1099 & iwadate & 0 &  \\ \hline 
33 & 奥行き画像センサを用いた展示空間の物理評価 & http://blog.shirai.la/wp-content/uploads/downloads/2012/11/VRSJ2011-ResBeK-F.pdf & 奥行き画像センサを用いた展示空間の物理評価 &  & 2012-11-29 07:11:00 & 1146 & iwadate & 0 &  \\ \hline 
34 & 全身運動を中心とした震災復興を伝えるシリアスゲームの開発 & http://blog.shirai.la/wp-content/uploads/downloads/2012/11/vrsj2011\_ver6.pdf & 全身運動を中心とした震災復興を伝えるシリアスゲームの開発 &  & 2012-11-29 07:14:00 & 983 & iwadate & 0 &  \\ \hline 
35 & 放射線可視化を通した科学コミュニケーション活動 & http://blog.shirai.la/wp-content/uploads/downloads/2012/11/NICOGRAPH2011\_kitada.pdf & 放射線可視化を通した科学コミュニケーション活動 &  & 2012-11-29 07:23:00 & 1154 & iwadate & 0 &  \\ \hline 
36 & 実世界指向ゲームインタラクション技術の歴史,フィロソフィ,そして近未来 & http://blog.shirai.la/wp-content/uploads/downloads/2012/11/ITE2011AS.pdf & 実世界指向ゲームインタラクション技術の歴史,フィロソフィ,そして近未来 &  & 2012-11-29 07:34:00 & 933 & iwadate & 0 &  \\ \hline 
37 & スマートフォンの高精細加速度センサを用いた抽象的動画作品視聴時のユーザ動作分析と作品のクオリティ向上手法の提案 & http://blog.shirai.la/wp-content/uploads/downloads/2012/11/NICOGRAPH2011Sp\_TakumiPaper.pdf & スマートフォンの高精細加速度センサを用いた抽象的動画作品視聴時のユーザ動作分析と作品のクオリティ向上手法の提案 &  & 2012-11-29 07:44:00 & 943 & iwadate & 0 &  \\ \hline 
38 & 感圧センサを使った感覚運動インタラクションのための自然なプレイヤ分析アルゴリズムと評価 & http://blog.shirai.la/wp-content/uploads/downloads/2012/11/NICOGRAPH\_LovePress.pdf & 感圧センサを使った感覚運動インタラクションのための自然なプレイヤ分析アルゴリズムと評価 &  & 2012-11-29 07:46:00 & 1177 & iwadate & 0 &  \\ \hline 
39 & クリーチャーデザインにおける、手書きに注目したデザインワークフローの提案 & http://blog.shirai.la/wp-content/uploads/downloads/2012/11/NICOGRAPH2011\_CreatureDesign\_NoriyukiYamamoto.pdf & クリーチャーデザインにおける、手書きに注目したデザインワークフローの提案 &  & 2012-11-29 07:48:00 & 1454 & iwadate & 0 &  \\ \hline 
40 & e-sports 映像のネット配信を考慮した速度変化演出の効果と特性 & http://blog.shirai.la/wp-content/uploads/downloads/2012/11/NICOGRAPH2011\_e-sports.pdf & e-sports 映像のネット配信を考慮した速度変化演出の効果と特性 &  & 2012-11-29 07:54:00 & 964 & iwadate & 0 &  \\ \hline 
41 & 放射能はオバケじゃない!霧箱制作 で学ぶ自然放射線の可視化 & http://blog.shirai.la/wp-content/uploads/downloads/2012/11/KAITSympo2011-Kiribako.pdf & 放射能はオバケじゃない!霧箱制作 で学ぶ自然放射線の可視化 &  & 2012-11-29 08:02:00 & 876 & iwadate & 0 &  \\ \hline 
42 & LovePress++: 物理入力に感応する新しい恋愛シリアスゲームの提案 & http://blog.shirai.la/wp-content/uploads/downloads/2012/11/Interaction2011\_Yamashita.pdf & LovePress++: 物理入力に感応する新しい恋愛シリアスゲームの提案 &  & 2012-11-29 08:08:00 & 730 & iwadate & 0 &  \\ \hline 
43 & 測域センサを用いた体験教育環境の物理的評価 & http://blog.shirai.la/wp-content/uploads/downloads/2012/11/ITSympo2011Iwadate.pdf & 測域センサを用いた体験教育環境の物理的評価 &  & 2012-11-29 08:15:00 & 898 & iwadate & 0 &  \\ \hline 
44 & 抽象的なアニメーション作品視聴に対する加速度センサを用いた自然なユーザ解析手法の提案 & http://blog.shirai.la/wp-content/uploads/downloads/2012/11/ITSympo2011Takumi.pdf & 抽象的なアニメーション作品視聴に対する加速度センサを用いた自然なユーザ解析手法の提案 &  & 2012-11-29 08:25:00 & 749 & iwadate & 0 &  \\ \hline 
45 & Development of serious game which use full body interaction and accumulated motion & http://blog.shirai.la/wp-content/uploads/downloads/2012/11/Gamic-nicograph.pdf & Development of serious game which use full body interaction and accumulated motion &  & 2012-11-29 08:35:00 & 905 & kitada & 0 &  \\ \hline 
46 & GAMIC: Exaggerated Real-Time Character Animation Control Method for Full-Body Gesture Interaction System & http://blog.shirai.la/wp-content/uploads/downloads/2012/11/Gamic-SIGGRAPH2011-F.pdf & GAMIC: Exaggerated Real-Time Character Animation Control Method for Full-Body Gesture Interaction System &  & 2012-11-29 08:43:00 & 1010 & kitada & 0 &  \\ \hline 
47 & SIGGRAPH 2011\_poster & http://blog.shirai.la/wp-content/uploads/downloads/2012/11/GAMICSIGPOS20110727.pdf & SIGGRAPH 2011\_poster &  & 2012-11-29 08:50:36 & 741 & kitada & 0 &  \\ \hline 
48 & Skeleton-based diverse creature design tool for mass production & http://blog.shirai.la/wp-content/uploads/downloads/2012/11/IREVA-SA2011F1p.pdf & Skeleton-based diverse creature design tool for mass production &  & 2012-11-29 08:56:00 & 1104 & kitada & 0 &  \\ \hline 
49 & SiggraphAsia2011\_Skelton\_Poster & http://blog.shirai.la/wp-content/uploads/downloads/2012/11/SiggraphAsia2011\_Skelton\_Poster.pdf & SiggraphAsia2011\_Skelton\_Poster &  & 2012-11-29 08:58:47 & 749 & kitada & 0 &  \\ \hline 
50 & Experimental methods and natural player analysis for sensory-motor interaction using pressure sensors & http://blog.shirai.la/wp-content/uploads/downloads/2012/11/NICOGRAPH2011International\_LovePress.pdf & Experimental methods and natural player analysis for sensory-motor interaction using pressure sensors &  & 2012-11-29 09:07:00 & 928 & kitada & 0 &  \\ \hline 
51 & ScritterHDR: Multiplex-Hidden Imaging on High Dynamic Range Projection & http://blog.shirai.la/wp-content/uploads/downloads/2012/11/ScritterHDRAbstract.pdf & ScritterHDR: Multiplex-Hidden Imaging on High Dynamic Range Projection &  & 2012-11-29 09:11:00 & 866 & kitada & 0 &  \\ \hline 
52 & ScritterHDR\_poster & http://blog.shirai.la/wp-content/uploads/downloads/2012/11/SIGGRAPH-ASIA-2011-POSTER.pdf & ScritterHDR\_poster &  & 2012-11-29 09:13:30 & 735 & kitada & 0 &  \\ \hline 
53 & Scritter-L: 時間停止機能と映像多重化システムを用いたe-sports イベントでの高い一体感を演出するバーチャル中継システムの提案 & http://blog.shirai.la/wp-content/uploads/downloads/2012/11/VRSJ2010-eSports.pdf & Scritter-L: 時間停止機能と映像多重化システムを用いたe-sports イベントでの高い一体感を演出するバーチャル中継システムの提案 &  & 2012-11-30 05:15:00 & 873 & iwadate & 0 &  \\ \hline 
54 & 摂動応答と重心動揺計を用いた嗜好画像のリアルタイム推定手法の提案 & http://blog.shirai.la/wp-content/uploads/downloads/2012/11/VRSJ2010-Hitomebore.pdf & 摂動応答と重心動揺計を用いた嗜好画像のリアルタイム推定手法の提案 &  & 2012-11-30 05:29:00 & 1067 & iwadate & 0 &  \\ \hline 
55 & ResBe:エンタテイメントシステム周囲のコミュニケーション場に対する遠隔評価手法の提案 & http://blog.shirai.la/wp-content/uploads/downloads/2012/11/VRSJ2010-ResBe.pdf & ResBe:エンタテイメントシステム周囲のコミュニケーション場に対する遠隔評価手法の提案 &  & 2012-11-30 05:31:00 & 875 & iwadate & 0 &  \\ \hline 
56 & 多重化映像表示における隠蔽映像生成アルゴリズム & http://blog.shirai.la/wp-content/uploads/downloads/2012/11/VRSJ2010-ScritterH.pdf & 多重化映像表示における隠蔽映像生成アルゴリズム &  & 2012-11-30 05:36:00 & 948 & iwadate & 0 &  \\ \hline 
57 & ステレオ立体視技術と高い互換性を持つ多重化映像提示システム およびコンテンツ制作手法の提案 & http://blog.shirai.la/wp-content/uploads/downloads/2012/11/VRSJ2010-Scritter.pdf & ステレオ立体視技術と高い互換性を持つ多重化映像提示システム およびコンテンツ制作手法の提案 &  & 2012-11-30 05:38:00 & 726 & iwadate & 0 &  \\ \hline 
58 & 測域センサを用いたResBeシステムとHeatmapによる実世界指向エンタテイメントシステムの物理評価手法 & http://blog.shirai.la/wp-content/uploads/downloads/2012/11/EC2010-ResBe0813F.pdf & 測域センサを用いたResBeシステムとHeatmapによる実世界指向エンタテイメントシステムの物理評価手法 &  & 2012-11-30 05:44:00 & 1329 & iwadate & 0 &  \\ \hline 
59 & A new "multiplex content"; displaying system compatible with current 3D projection technology & http://blog.shirai.la/wp-content/uploads/downloads/2012/11/a79-nagano.pdf & A new "multiplex content"; displaying system compatible with current 3D projection technology &  & 2012-11-30 08:41:00 & 1046 & kitada & 0 &  \\ \hline 
60 & SIGGRAPH 2010\_poster & http://blog.shirai.la/wp-content/uploads/downloads/2012/11/SIGGRAPH-POSTERvr6.4.pdf & SIGGRAPH 2010\_poster &  & 2012-11-30 08:45:04 & 789 & kitada & 0 &  \\ \hline 
61 & Scritter: A multiplexed image system for a public screen & http://blog.shirai.la/wp-content/uploads/downloads/2012/11/Scritter-ReVolution2010.pdf & Scritter: A multiplexed image system for a public screen &  & 2012-11-30 08:54:00 & 942 & kitada & 0 &  \\ \hline 
62 & 日仏VRにおける『面白い』研究スタイルの相違 & http://blog.shirai.la/wp-content/uploads/downloads/2012/11/VRSJ2012-FrJpOSAki.pdf & 日仏VRにおける『面白い』研究スタイルの相違 &  & 2012-11-30 09:04:50 & 0 & kitada & 0 &  \\ \hline 
63 & Wiiリモコンとエンタテイメント技術の新学習指導要領への活用 & http://blog.shirai.la/wp-content/uploads/downloads/2012/11/WiiRemoteWS-Odawara20110822Web.pdf & Wiiリモコンとエンタテイメント技術の新学習指導要領への活用 &  & 2012-11-30 09:14:00 & 0 & kitada & 0 &  \\ \hline 
64 & 3Dディスプレイ互換の多重化映像システム & http://blog.shirai.la/wp-content/uploads/downloads/2012/11/TSY2011-ScritterS.pdf & 3Dディスプレイ互換の多重化映像システム &  & 2012-11-30 09:42:37 & 0 & kitada & 0 &  \\ \hline 
65 & LovePressOpenBeta2\_shiori. & http://blog.shirai.la/wp-content/uploads/downloads/2012/12/LovePressOpenBeta2\_shiori.zip & LovePressOpenBeta2\_shiori. &  & 2012-12-04 05:46:19 & 459 & kitada & 0 &  \\ \hline 
66 & LovePressOpenBeta2\_yuko & http://blog.shirai.la/wp-content/uploads/downloads/2012/12/LovePressOpenBeta2\_yuko.zip & LovePressOpenBeta2\_yuko &  & 2012-12-04 05:48:49 & 831 & kitada & 0 &  \\ \hline 
67 & LovePressOpenBeta2\_suzuka & http://blog.shirai.la/wp-content/uploads/downloads/2012/12/LovePressOpenBeta2\_suzuka.zip & LovePressOpenBeta2\_suzuka &  & 2012-12-04 05:52:09 & 501 & kitada & 0 &  \\ \hline 
68 & LCDプロジェクターを用いた偏光によるステレオ立体視のための画質向上アルゴリズム & http://blog.shirai.la/wp-content/uploads/downloads/2013/01/IPSJ2012-Koide.pdf & LCDプロジェクターを用いた偏光によるステレオ立体視のための画質向上アルゴリズム &  & 2013-01-18 11:56:19 & 1598 & admin & 0 &  \\ \hline 
69 & 実世界指向エンタテイメントシステムのための 失敗しないゲームデザインの提案 & http://blog.shirai.la/wp-content/uploads/downloads/2013/02/106a039f0598184d9c03cdda599b8b22.pdf & 2012年度卒業論文要旨\_奈良 &  & 2013-02-19 18:31:57 & 757 & kitada & 0 &  \\ \hline 
70 & ソーシャルゲームのジャーナリズム構築を 目的としたブログ自動生成 & http://blog.shirai.la/wp-content/uploads/downloads/2013/02/ccd9f68c19e478335a9253eadd609eae.pdf & 2012年度卒業論文要旨\_亀山 &  & 2013-02-19 09:46:20 & 777 & kitada & 0 &  \\ \hline 
71 & スマートフォンの加速度センサによる“笑いのツボ”の可視化 & http://blog.shirai.la/wp-content/uploads/downloads/2013/02/8f3059e28e6707c08f92105ec3cd9fef.pdf & 2012年度卒業論文要旨\_鈴木 &  & 2013-02-19 09:49:10 & 969 & kitada & 0 &  \\ \hline 
72 & コンピューターゲームジャンル変容と独立性の可視化 & http://blog.shirai.la/wp-content/uploads/downloads/2013/02/899256d7c138004ebf482389875b2023.pdf & 2012年度卒業論文要旨\_加藤航 &  & 2013-02-19 09:53:32 & 1055 & kitada & 0 &  \\ \hline 
73 & センサデータを用いた ユーザ体験空間における行動評価 & http://blog.shirai.la/wp-content/uploads/downloads/2013/02/1f382bcd3228f32ed300231ad9cd9683.pdf & 修士論文要旨\_岩楯 &  & 2013-02-19 09:57:47 & 807 & kitada & 0 &  \\ \hline 
74 & 科学館向けエンタテイメントシステムにおける成績データを用いたユーザー分析 & http://blog.shirai.la/wp-content/uploads/downloads/2013/04/AS2013\_kitada.pdf & 本稿では,2012年7月21日〜9月2日に新潟県立自然科学館で開催された企画展「謎解きアドベンチャー失われた紋章」で使用したRFIDとデータベースサーバ,プロジェクションマッピングを用いた多人数が同時に参加できる科学クイズエンタテイメントシステムにおいて,サーバー側に記録されている体験者の成績データ群を分析し,体験者の傾向について報告をおこなう.従来の科学館向けエンタテイメントシステムの報告・調査では明らかにされてこなかった,体験者の科学理解の状況と紙メディアを使った具体的なフィードバック方法の提案をおこなう.

This article issues a case study about tendency of the experient in a science museum exhibition, "Science Quest" project which had been held in Niigata Science Museum from July 21, 2012 to September 2. The exhibition was physical human scale interactive system which uses science quiz, RFID and projection mapping to tell elemental science as a special exhibition in a summer holiday. Through a player analysis using post data from answered database of the experients, it reports a state of experients' science understanding and difficulty in vocal interaction devices. Then it proposes actual method using paper media for a better science understanding. &  & 2013-04-01 13:15:00 & 829 & shirai & 0 &  \\ \hline 
75 & 瞬刊少年マルマル & http://blog.shirai.la/wp-content/uploads/downloads/2013/05/AS2013\_manga.pdf & 瞬刊少年マルマル &  & 2013-05-22 18:24:30 & 855 & yuto & 0 &  \\ \hline 
76 & Manga Generator: Immersive Posing Role Playing Game in Manga World & http://blog.shirai.la/wp-content/uploads/downloads/2013/05/Manga-VRIC2013f.pdf & Manga Generator:Immersive Posing Role Playing Game in Manga World &  & 2013-05-22 18:39:43 & 2554 & yuto & 0 &  \\ \hline 
77 & スマートフォンの加速度センサを用いた微小不随意運動検出による動画視聴時の笑い評価手法 & http://blog.shirai.la/wp-content/uploads/downloads/2013/09/VRSJ2013\_kitada.pdf &  &  & 2013-09-13 20:26:00 & 1362 & shirai & 0 &  \\ \hline 
78 & マンガ没入型 VR エンタテイメントシステムにおけるコンテンツ制作手法 & http://blog.shirai.la/wp-content/uploads/downloads/2013/09/VRSJ2013\_Koide.pdf & 本論文は第 20 回国際学生対抗バーチャルリアリティコンテストを機会に開発を行った,マン ガの中に入り込んで体験者のオリジナルマンガを作成するマンガ没入型エンタテイメントシステム 『瞬刊少年マルマル』および,その後の研究である『Manga Generator』における追加・改善事項,コ ンテンツ開発のワークフロー,それを用いて行った産業向け展示の手法について報告を行う.あわ せて今後のマンガ没入型エンタテイメントシステム展示における課題と可能性について述べる. キーワード: マンガ没入型,エンタテイメント,VR,コンテンツ制作手法,展示開発 &  & 2013-09-13 20:39:00 & 1082 & shirai & 0 &  \\ \hline 
79 & VR エンタテイメントシステムのための リアルタイムマンガ風画像生成シェーダーの開発 & http://blog.shirai.la/wp-content/uploads/downloads/2013/09/VRSJ2013\_fujumura.pdf &  &  & 2013-09-13 20:43:00 & 1237 & shirai & 0 &  \\ \hline 
80 & WebSocket を用いたスマートフォン上での エンタテイメントコンテンツ閲覧時のリアルタイム行動分析 & https://ipsj.ixsq.nii.ac.jp/ej/index.php?active\_action=repository\_view\_main\_item\_detail\&item\_id=95983\&item\_no=1\&page\_id=13\&block\_id=8 & 論文抄録:	概要:スマートフォン上でのエンタテイメントコンテンツ閲覧時のリアルタイム行動分析手法について報告する.「楽しみ」は主観的な人間の感情であるが,「笑い」は視聴者の不随意運動として,スマートフォンの高分解能加速度センサにて得ることが可能である.スマートフォン上での映像視聴時に起きる自然な不随意運動を,スマートフォン上の高分解能加速度センサを用いて取得し,視聴者の主観的な情動を統計的かつ集合的に分類取得するアルゴリズムおよびアプリケーション開発を行っている.実験システムは,データ収集のためにRuby とSinatra フレームワークによって開発されており,将来的な実装として,WebSocket のサーバーサイドJavaScript での実装であるNode.js を用いて,Web ベースの分散非同期接続での実装可能性を検討し,LAN/WAN 環境においてベンチマークを行った.推定アルゴリズムは,ムービープレイヤーアプリケーション「L-PoD」に実装され,同一のエンタテイメント・コンテンツにおける被験者間の笑いの特性の違いをProperty of Difference (PoD) として視覚化することができる. 
論文抄録(英):	Abstract: This article contributes to report a method to classify human laughing during video watching using Smartphone. Fun is a subjective human emotion but Laughing is possible to obtain as audience's involuntary movement. The algorithm have been implemented into a movie player application L-PoD" and it can visualize subjects' Property of Difference (PoD) of laughing part during a same comedy content. For the data collection, the experimental system has been developed by Ruby and Sinatra framework and it has been compared with WebSocket imprementation in server side JavaScript Node.js", as a further implementation. &  & 2013-10-16 14:35:00 & 978 & shirai & 0 & http://blog.shirai.la/wp-content/uploads/downloads/2013/10/EC2013-ASOS\_kitada.pdf \\ \hline 
81 & スマートフォンを用いた実世界指向パーティゲームの提案 & https://ipsj.ixsq.nii.ac.jp/ej/?action=pages\_view\_main\&active\_action=repository\_view\_main\_item\_detail\&item\_id=95980\&item\_no=1\&page\_id=13\&block\_id=8 & 論文抄録:	本論文では,スマートフォンを用いた実世界指向エンタテイメントアプリケーションとして製作した「王様スロット」について報告する.「王様スロット」はパーティゲームにおいて,iPhone がプレイヤーにさまざまなアクションを要求するスロットマシンであり,アクションは「頭を下げる」,「精神統一」,「ガッツポーズ」,「ビンタされる」,「隣の人とツーショット」がある.本報告ではこれらのアクションを実装する方法について報告する. 
論文抄録(英):	This paper reports about a case study of game system design for a physical world oriented entertainment applications using smartphone, which have been named as "King Slots"."King slot" is a slot machine for party game. The application requires a variety of actions to the players. It reports how to implement player action recognitions like "Bowing", "Concentration", "Victory Pose", "Slapping" and "Taking a photo with your neighbor". &  & 2013-10-19 16:56:00 & 1090 & shirai & 0 & http://blog.shirai.la/wp-content/uploads/downloads/2013/10/ec2013ishikawa.pdf \\ \hline 
82 & エンタテイメントシステム展示を対象とした質的評価ツールの提案 & https://ipsj.ixsq.nii.ac.jp/ej/index.php?active\_action=repository\_view\_main\_item\_detail\&item\_id=95981\&item\_no=1\&page\_id=13\&block\_id=8 & 論文抄録:	この論文は,エンタテイメントシステム展示のための質的評価ツールについて報告する.伝統的に,エンタテイメントシステムの評価は標準的な方法が確立されていない.通常は,体験の新規性やコンテンツの売り上げ,またはアンケートにより評価される.しかし展示会などにおける新規のエンタテイメントシステムを評価するために,体験者のみならず,体験しなかった訪問者も含め,様々な属性を含んで評価されるべきであろう.本研究では、公共実験において,インタラクティブ作品を体験する前のユーザに焦点をあて,幅広いユーザが使える質的評価ツールを構築することに挑戦している.Kinect のデバイスとそのAPI は,訪問者の混雑レベルを可視化することが可能であり,グループまたは単独の訪問者を分類することに成功した. 
論文抄録(英):	This article reports about qualitative analysis tool for entertainment system exhibition. Traditionally, evaluation of the entertainment system is not established as a standard method. Normally, it is evaluated by content sales and/or questionnaire as the novelty of the experience more often. However, in order to evaluate novel entertainment system in an exhibition, it should be evaluated including various properties that not only the tried players but also untried audiences. In this research, we had installed Kinect in previous step of the main experiment of the tested entertainment systems in a public experiment. It focuses on the user prior to experience the entertainment system to build a measurement tool of interaction for various experimenters. The Kinect device and its API can be possible to visualize crowd level of visitor and it can be use as a classifier of a group or solitary visitor. &  & 2013-10-19 17:04:00 & 863 & shirai & 0 & http://blog.shirai.la/wp-content/uploads/downloads/2013/10/EC2013Tadokoro9.pdf \\ \hline 
83 & 年齢層とゲーミングデバイスの違いによる面白さの比較調査 & https://ipsj.ixsq.nii.ac.jp/ej/index.php?active\_action=repository\_view\_main\_item\_detail\&item\_id=95982\&item\_no=1\&page\_id=13\&block\_id=8 & 論文抄録:	エンタテイメントシステム,特にインタラクティブな体験要素を持ったシステムの面白さとは「面白そう」という第一印象のインパクトと一致しないことは経験からも理解できる.一般には「ユーザビリティ」や「興味」のきっかけとなった「誘引性」が「面白さ」という感想に直接寄与していると考えられる.本研究において,同一コンテンツを複数のデバイスの誘引性・ユーザビリティ・インタレストが体験後の「面白さ」に直接の原因とならず,特に13~15歳以外の体験者層にとっては,ユーザビリティの高いデバイスが面白いエンタテイメントシステムとして選択されないことを発見した. 
論文抄録(英):	This paper contribute to share an evaluation technique to define amusingness (Omoshirosa).Amusingness does not match the impact of the first impression from the experience. Interactive system is believed to have contributed directly to the impression that triggered the "interest", "usability" and "attractivity" in general. In this study, it had been experimented before-after exhibited on three devices with a same content by forced choice method questionnaire. As a highly interesting result,interest,usability and attractivity did not match the direct cause of the amusingness after the experience. Exclude 13-15 years old people, the device which has chosen as a best usability was not selected as the best experience in total. &  & 2013-10-19 17:06:00 & 787 & shirai & 0 & http://blog.shirai.la/wp-content/uploads/downloads/2013/10/EC2013\_kunitomi\_final.pdf \\ \hline 
84 & 姿勢評価によるリアルタイム感情推定を特徴とする動的マンガ生成システム「Manga Generator」 & http://blog.shirai.la/wp-content/uploads/downloads/2013/10/CEDEC2013\_manga\_web.pdf & 「Manga Generator」は、体験者が少年マンガの主人公となり、ストーリーに沿ったポーズを取ることで、マンガを自動生成するエンタテイメントシステムである。生成されたマンガは印刷され、ユーザが持ち帰ることができる。マンガを構成する要素である吹き出しセリフ、効果線、擬音といったマンガ効果の付加は自動で行い、姿勢評価を用いたリアルタイム感情選択によって、体験者のポーズに沿ったマンガ効果を自動で付加する。一般的なゲームシステムと異なり、ストーリーの設計はコントロール可能である。また、通常は「一期一会」となりがちである体感型エンタテイメントシステムと異なり、SNSとの連携によりスマートフォンアプリや広告メディアとしての可能性を持っている。
http://cedec.cesa.or.jp/2013/program/GD/7225.html &  & 2013-10-19 17:51:00 & 746 & shirai & 0 &  \\ \hline 
85 & 自由なレイアウトとリフロー機能を備えたハイブリッド電子書籍の提案 & http://blog.shirai.la/wp-content/uploads/downloads/2013/10/nico2013Lee.pdf & 現在,爆発的に普及しつつある電子書籍のフォーマットには,PDF をベースとしたマンガや雑誌等レイアウトを重視したものと,HTML をベースとした新聞や小説など文章,内容を重視したもので,大きく分けて 2 つの形式が存在する.本研究ではそのどちらにも着目し,レイアウトを崩さず且つ文章を綺麗に表現することを主軸に,学会誌,マンガ,小説の 3 つの点からリフローを利用したハイブリッド電子書籍の提案を行う. &  & 2013-10-19 18:20:23 & 698 & shirai & 0 &  \\ \hline 
86 &  SD法を用いた電子ペーパーデバイスと紙媒体におけるフォント比較評価 & http://blog.shirai.la/wp-content/uploads/downloads/2013/10/nico2013-koike.pdf & 近年,電子出版の発展により手軽に自筆の書籍を出版できるようになった.しかし出版にあたって,種類が多岐にわたるフォントの中から適切なフォントを選択して使用するのは困難である. 本報告は, フォントが与えるイメージを特定することを目的に, 電子ペーパーデバイス上におけるフォントのデザインに対する心理的評価を,SD 法を用いて行い,従来の紙メディアとの特性を比較する.本研究では特に横太ゴシック体,ゴシック体,教科書体の評価に関して E Ink と紙との結果に違いが確認された &  & 2013-10-19 18:25:00 & 1205 & shirai & 0 &  \\ \hline 
87 & KinEmotion: Context controllable emotional motion analysis method for interactive cartoon generator & http://dl.acm.org/citation.cfm?id=2503467 & Abstract: Recently, cartoon contents are applying to various media like interactive systems. In a near future, the desire of user may become to immerse their live experience into a cartoon content deeply. In automatic cartoon generation environment using NUI (Natural User Interface) like Kinect, we can comprehend the importance of linking emotion expression with user's posture. Its story and impression are uncontrollable, if the system could not choose a suitable effect for each user motion. The system should have story driven method to protect the interpretation of the world, even if there is its original piece of manga. &  & 2013-10-19 18:31:22 & 822 & shirai & 0 & http://blog.shirai.la/wp-content/uploads/downloads/2013/10/c103-f103\_3992-a14-final\_abstract-v2.pdf \\ \hline 
88 & KinEmotion: Context-Controllable Motion Analysis Method for Interactive Cartoon Generator & http://blog.shirai.la/wp-content/uploads/downloads/2013/10/SIGGRAPH2013MangaPoster.pdf &  &  & 2013-10-19 18:49:31 & 679 & shirai & 0 &  \\ \hline 
89 & マンガ没入型エンタテイメントシステムの可能性 & http://blog.shirai.la/wp-content/uploads/downloads/2014/08/HCG2013-C-3-3.pdf & あらまし  本論文は,マンガ没入型 VR エンタテイメントシステム『瞬刊少年マルマル(MangaGenerator)』について,エンタテイメントシステムとしての体験設計と展示を通した評価について報告を行うものである.システムで使用したキネマティクスを用いた感情推定手法『KinEmotion』について,2 種類のアンケートによる姿勢と背景画像のマッチングについて評価を行い,ユーザの感情認識と表現したい感情認識が概ね一致することを確認した.
キーワード  マンガ,仮想現実,Kinect,感情表現 &  & 2013-10-22 18:27:00 & 679 & shirai & 0 &  \\ \hline 
90 & 博物館ネットワーク事業:相模原市立博物館にはどんな人が来ているか	 & http://blog.shirai.la/wp-content/uploads/downloads/2013/11/Mukai-Museum-20131108.pdf & 相模原市立博物館と白井研究室の協働事業である「みんなでつくる相模原『知的探求散策アルバム』」は,市立博物館が計画す
る「さがみはらどこでも博物館」という地域の博物館施設・旧跡をつなぎ,市民に展開する事業である.平成25年度は初年度であり,
民俗探訪会に参加し,取材で得られたデータのコンテンツ化を行った.また,現在の相模原市立博物館における来場者約1,850
人に対して大規模なアンケートを実施し,注目する世代間における博物館に対する期待と現状を調査し,そして,サイネージの展
示でコンテンツの展開を行った.今回の活動を通して,民俗探訪会では参加者の意見や地域の文化,歴史をコンテンツ化できる
ことを知り,アンケート調査では来館者の年齢層,博物館への期待度などを知ることができた.また,サイネージの展開では今後
の課題と展開方法を模索することができた. &  & 2013-11-15 00:30:56 & 1166 & shirai & 0 &  \\ \hline 
91 & 学生VRコンテストを起点としたVRエンタテイメントシステム開発とその報告 & http://blog.shirai.la/wp-content/uploads/downloads/2013/12/Oukan20131018.pdf & Abstract- 本稿では,第20回国際学生対抗バーチャルリアリティコンテストを起点に開発を行ったVRエンタテイメントシステム『瞬刊少年マルマル/MangaGenerator』のとその後の研究の推移,産業化事例について報告する.
Index terms- Manga , VR , Kinect , Emotional expressions &  & 2013-12-21 20:33:00 & 811 & shirai & 0 &  \\ \hline 
92 & 相模原市立博物館 来館者アンケート-設問 (H25年7-8月実施) & http://blog.shirai.la/wp-content/uploads/downloads/2014/01/anke-20130715a2.pdf & 相模原市⽴博物館情報ネットワークセンター事業
みんなでつくる相模原
「知的探求散策アルバム」
平成25年7月実施 &  & 2014-01-09 23:30:00 & 2124 & shirai & 0 &  \\ \hline 
93 & MangaGeneratorMediaKit & https://www.dropbox.com/s/bnok23laqprko73/MangaGeneratorMediaKit.zip & Manga Generator Media Kit
 http://j.mp/MangaOtoiawase & 1.00 & 2014-02-12 05:58:00 & 3 & shirai & 0 &  \\ \hline 
94 & Labo PowerPoint Template & http://blog.shirai.la/wp-content/uploads/downloads/2014/03/ShiraiLab.potx\_.zip & PowerPoint Template (potx) & 2014 & 2014-03-02 17:34:00 & 3 & shirai & 1 & https://www.dropbox.com/s/owrozlrefuol053/ShiraiLab.potx \\ \hline 
95 & 芸術科学会誌「DiVA」第33号(2013年夏号) & http://j.mp/ASdiva33 & 第33号(2013年夏号)
■巻頭言(白井暁彦)
■芸術科学フォーラム2013報告(藤代一成)
■芸術科学会研究セミナー報告(宮田一乘)
■連載:芸術科学のウラオモテ
■連載:研究室リレー訪問
■連載:海外だより
■学会便り/これからの予定/読者アンケート &  & 2014-03-05 08:23:00 & 788 & shirai & 0 &  \\ \hline 
96 & 芸術科学会誌「DiVA」第34号(2013年冬号) & http://j.mp/ASdiva34 & 第34号(2013年冬号)
■巻頭言(西原清一)
■NICO Int. 2013報告(鶴野玲治)
■EC2013報告
(白井暁彦, 伊藤貴之, 宮田一乘)
■連載:DiVAギャラリー
■連載:研究室リレー訪問
■連載:芸術科学のウラオモテ
■連載:研究セミナー開催報告
   (長尾将宏)
■連載:「アート\&テクノロジー東北2013」
 開催報告(今野晃市)
■連載:海外だより
■連載:論文ダイジェスト
■支部便り/既刊DiVA &  & 2014-03-05 08:26:00 & 686 & shirai & 0 &  \\ \hline 
97 & 「未来のゲームセンター in アミューあつぎグランドオープンイベント」フライヤー & http://blog.shirai.la/wp-content/uploads/downloads/2014/04/Amyu\_PosterForAmyu\_vLight.pdf &  &  & 2014-04-17 16:31:00 & 376 & shirai & 0 &  \\ \hline 
98 & 「未来のゲームセンター in アミューあつぎグランドオープンイベント」プレスリリース & http://blog.shirai.la/wp-content/uploads/downloads/2014/04/Press20140416Amyu.pdf & 神奈川工科大学 情報メディア学科 学生有志が
~厚木市制60周年記念カウントダウン事業~「アミューあつぎ」オープニングイベントにて「未来のゲームセンター」を発表(2014/4/16) &  & 2014-04-17 16:39:56 & 360 & shirai & 0 &  \\ \hline 
99 & 神奈川工科大学 情報メディア学科 白井研究室の研究成果2点が先端技術館@tepiaにおいて常設展示化 & http://blog.shirai.la/wp-content/uploads/downloads/2014/04/PressKit-TEPIA.zip & 神奈川工科大学 情報メディア学科 白井研究室の研究成果2点が先端技術館@tepiaにおいて常設展示化
■
2014年4月22日(火)にリニューアルオープンする「先端技術館@tepia」(一般財団法人 高度技術社会推進協会・東京都港区北青山)にて、神奈川工科大学 情報学部 情報メディア学科 白井研究室の研究成果を中心とした2点が常設展示として「テクノロジースタジオ」に長期展示公開されます。
 出展される研究成果2点は、国際学生VRコンテスト(IVRC2012)で高い評価を受け、その後研究として世界的に注目されている“マンガ没入型”エンタテイメントシステム「マンガジェネレーター/瞬刊少年マルマル」と、本学・谷中一寿教授らと開発した3D互換の多重化映像技術「Scritter(スクリッター)」です。2つの展示は今後1年間の長期展示を予定されており、展示を通して白井研究室のテーマである「エンタテイメントシステム」と「人間と面白さ」について、本学・情報工学科の大塚真吾准教授と共同し、実験フィールドとして来場者の振る舞いデータを非接触で取得し分析します。
我が国を代表する先端技術が集まる館における、展示される親しみやすい展示です。みなさまのご来館・ご体験をお待ちしております。 &  & 2014-04-18 22:53:00 & 472 & shirai & 0 &  \\ \hline 
100 & プレスリリース:テレビの多重化を可能にする汎用ソフトウェア「ExPixel」の開発に成功 & http://blog.shirai.la/wp-content/uploads/downloads/2014/05/ExPixel-PressRelease20140502.zip & プレスリリース:2014年5月2日(金)
神奈川工科大学 情報学部 情報メディア学科 白井暁彦准教授は、従来不可能な技術であった直視型ディスプレイにおける映像多重化と裸眼による隠蔽画像を実現するソフトウェア技術「ExPixel」の開発に成功した。フランスと日本国内で一般向け公開実験を実施し好評を得た。 &  & 2014-05-02 15:42:25 & 381 & shirai & 0 &  \\ \hline 
101 & FamiLinkTV: Expanding the Social Value of the Living Room with Multiplex Imaging Technology & http://blog.shirai.la/wp-content/uploads/downloads/2014/07/revolution2014\_submission3.pdf & Hisataka Suzuki, Yannick Littfass, Rex Hsieh, Hiroki Taguchi Fujimura
Wataru, Yukua Koide, Akihiko Shirai,
FamiLinkTV: Expanding the Social Value of the Living Room with Multiplex Imaging Technology,
Virtual Reality International Conference - Laval Virtual 2014 Proceedings &  & 2014-05-16 03:44:00 & 2147 & hisataka & 0 & https://www.dropbox.com/s/iz1ykuck6m43k36/FamiLinkTV-FinalRC2.pdf \\ \hline 
102 & “Scritter” to “1p2x3D”: application development using multiplex hiding imaging technology & http://blog.shirai.la/wp-content/uploads/downloads/2014/07/vric2014\_submission\_501.pdf & Yannick Littfass, Hisataka Suzuki, Yukua Koide, Akihiko Shirai,
“Scritter” to “1p2x3D”: application development using multiplex hiding imaging technology
Virtual Reality International Conference - Laval Virtual 2014 Proceedings &  & 2014-05-16 04:06:00 & 593 & hisataka & 0 &  \\ \hline 
104 & 【富士通SSL】PRESS RELEASE多重化不可視映像技術の実用化に着手 & http://blog.shirai.la/wp-content/uploads/downloads/2014/07/【富士通SSL】PRESS-RELEASE\_多重化不可視映像技術の実用化に着手.pdf & 新たなエクスペリエンスを提供する新商品の創出を目的として エンターテインメントシステムのエヴァンジェリスト白井暁彦准教授と産学共同研究を開始
-多重化・不可視映像技術の実用化に着手-
株式会社富士通ソーシアルサイエンスラボラトリ(本社:川崎市中原区小杉町、代表取締役社長:川口浩幸、以下: 富士通 SSL)は、ICT を活用することで新たなエクスペリエンスを提供する商品の開発を加速するため、エンターテイ ンメントシステム(注1)の第一人者である神奈川工科大学 情報学部 情報メディア学科(所在地:神奈川県厚木市)の 白井暁彦准教授と産学共同研究を開始し、8月1日より多重化・不可視映像技術の実用システム開発に着手します。 &  & 2014-07-31 17:27:36 & 518 & shirai & 0 &  \\ \hline 
105 & ExPixel: PixelShader for multiplex-image hiding in consumer 3D flat panels (Poster) & http://blog.shirai.la/wp-content/uploads/downloads/2014/08/SIGGRAPH2014ExPixelPoster09.pdf &  &  & 2014-08-14 01:33:32 & 1069 & hisataka & 0 &  \\ \hline 
106 & ExPixel: PixelShader for multiplex-image hiding in consumer 3D flat panels & http://dl.acm.org/citation.cfm?id=2614217.2633393 &  &  & 2014-08-14 01:36:00 & 900 & hisataka & 0 &  \\ \hline 
107 & 直線偏光による多重化隠蔽型ハイブリッド3Dディスプレイにおける画質評価 & http://ci.nii.ac.jp/naid/110009809731 & 多重化隠蔽映像技術によって,メガネの切り替えのみで単一のディスプレイを2D と3D ディスプレイのハ
イブリッドディスプレイとして使用することができる「2x3D」技術において,複数の被験者における画質向上のため
の評価実験を行った. &  & 2014-09-14 12:01:00 & 560 & shirai & 0 & http://blog.shirai.la/wp-content/uploads/downloads/2014/09/3DIT2014-FujimuraFinal.pdf \\ \hline 
108 & 液晶フラットパネルにおける多重化隠蔽映像の試行と実現 & http://blog.shirai.la/wp-content/uploads/downloads/2014/09/3DIT2014-Koide.pdf & 本研究は,映像技術の新しい付加価値創出として開発を行ってきた,3D 互換の映像多重化技術の,液晶フラットパネル上での実現について述べている.普及型の3D ディスプレイに用いられるライン・バイ・ライン方式を利用した映像の多重化と隠蔽について試作し,その結果から液晶フラットパネルを用いた多重化隠蔽映像の実現を行った. &  & 2014-09-14 15:20:00 & 669 & shirai & 0 &  \\ \hline 
109 & HMD装着時における首によるジェスチャ認識 ~ 首可動域の特性 ~ & http://ci.nii.ac.jp/naid/110009807451 & 本研究ではHMDのセンサフュージョンをそのまま利用し,普及しているコンテンツ開発環境であるUnity3Dにおける首の動きを利用したジェスチャ入力の認識方法を提案する.最も基本的な方法として,絶対的な回転角度によって「肯定」,「否定」,「疑問」の3つのジェスチャを認識する.
Abstract: This article proposes a motion recognition method for Oculus Rift, which is currently the most popular head mounted display available, by using neck motion as non-verbal interaction. We use sensor-fusion, which is already equipped on the Oculus Rift, to detect user’s neck motion. The method can define three gestures, namely, “agree,” “disagree,” and “make a question,” by determining the constancy of the head turning angle.
Keyword; Oculus Rift,head tracking,,natural user interface,virtual reality. &  & 2014-09-14 15:51:00 & 859 & shirai & 0 & http://blog.shirai.la/wp-content/uploads/downloads/2014/09/ITE2014-TaguchiAS.pdf \\ \hline 
110 & 自作アーケードゲーム「アオモリズム」開発を通じた エンタテインメントシステム開発教育の実践 & http://blog.shirai.la/wp-content/uploads/downloads/2014/09/N1自作アーケードゲーム「アオモリズム」開発を通じた-エンタテインメントシステム開発教育の実践.pdf & 近年の大学、専門学校での教育目的でのゲーム制作はゲームソフトウェアの制作が主になっている現状があるが、神奈川工科大学ではゲームを「エンタテインメントシステム」と捉え、一つのプレイヤー体験のために入出力デバイスからハードウェア、ソフトウェア全てのシステムを設計、実現するまでのプロセスを教員指導の下学生に体験させる事を行いオリジナルのアーケードゲーム「アオモリズム」を完成させた。途中に起きた問題、その解決のプロセスにおける発見を参加した学生と指導した教員の視点で伝える。 &  & 2014-09-14 16:02:21 & 656 & shirai & 0 &  \\ \hline 
111 & 3Dディスプレイに付加価値を与える多重化隠蔽映像技術 & http://blog.shirai.la/wp-content/uploads/downloads/2014/09/S23Dディスプレイに付加価値を与える多重化隠蔽映像技術.pdf & 白井 暁彦 &  & 2014-09-14 16:18:00 & 1 & shirai & 1 &  \\ \hline 
112 & 測域センサデータの可視化 & http://www.ieice.org/ken/paper/20131218BB6s/ & コミュニケーションロボットやデジタルサイネージといったデジタルコンテンツを体験しているユーザに対して物理的な評価を行うことで,システムの質的向上や工学的なコンテンツ開発ノウハウを蓄積できる.そこで非接触でユーザ属性を測ることが可能な測域センサを用いてユーザの動作を可視化した. 
Abstract: Recently, a trial to accumulate know-how of the qualitative improvement of the system and the engineering-like contents development using physical evaluation is carried out for users who experienced digital content such as communication robot and digital signage. Therefore we visualized the users’ behavior using the scanning laser rangefinder that could measure positional information of the user by non-contact. &  & 2014-09-14 16:51:00 & 519 & shirai & 0 &  \\ \hline 
113 & Neck gesture recognition by using constancy of head turning & http://blog.shirai.la/wp-content/uploads/downloads/2014/09/VRIC\_ACM\_NeckInteractionAS3.pdf & ABSTRACT: This article proposes a motion recognition method for Oculus Rift, which is currently the most popular head mounted display
available, by using neck motion as non-verbal interaction. The method can define three gestures, namely, “agree,” “disagree,” and “make a question,” by determining the constancy of the head turning angle. &  & 2014-09-14 17:44:53 & 827 & shirai & 0 &  \\ \hline 
114 & Neck gesture recognition by using constancy of head turning (poster) & http://blog.shirai.la/wp-content/uploads/downloads/2014/09/Neck-gesture-recognition.pdf &  &  & 2014-09-14 17:50:05 & 830 & shirai & 0 &  \\ \hline 
115 & 博物館ネットワーク事業:相模原市立博物館の来館者調査 & http://sagamiharacitymuseum.jp/wp-content/uploads/2014/07/6037a3a7c5e62613ccc35e56cdf1b4eb.pdf &  &  & 2014-09-16 02:46:00 & 755 & shirai & 0 &  \\ \hline 
116 & 球体ディスプレイとモーション入力を用いた科学コンテンツのエンタテイメントシステム化と展示評価手法 & https://ipsj.ixsq.nii.ac.jp/ej/?action=repository\_uri\&amp;item\_id=102966\&amp;file\_id=1\&amp;file\_no=1 &  &  & 2014-10-01 23:01:03 & 659 & shirai & 0 &  \\ \hline 
117 & 多重化クリエイソン\#01,02フライヤー & http://shirai.bf1.jp/Pub/BigFileForWeb/MultiplexCreathonF02out.pdf &  & v.1 & 2014-10-18 00:26:00 & 255 & shirai & 0 &  \\ \hline 
118 & 相模Ingress部 部員手帳 & http://blog.shirai.la/wp-content/uploads/downloads/2014/11/Sagamingressbu.pdf & 相模Ingress部 2014/11/14発行 & v.1.00 & 2014-11-14 09:17:22 & 306 & shirai & 0 &  \\ \hline 
119 & ゲームの次元を拡張する次世代多重化映像技術「ExPixel」 & http://cedil.cesa.or.jp/session/detail/1234 &  &  & 2015-01-13 19:15:24 & 559 & hisataka & 0 &  \\ \hline 
120 & 多重化不可視映像技術(第2報)-FPGAを用いたハードウェア化- & http://blog.shirai.la/wp-content/uploads/downloads/2015/01/SIS2014-79多重化不可視映像技術(第2報)\_―FPGAを用いたハードウェア化―.pdf &  &  & 2015-01-13 20:38:30 & 674 & hisataka & 0 &  \\ \hline 
121 & マンガ没入型エンタテイメントシステムの国際メディア発信ツールとしての進化 & http://blog.shirai.la/wp-content/uploads/downloads/2015/01/HCG2014-B-6-4.pdf &  &  & 2015-01-13 20:42:00 & 571 & hisataka & 0 &  \\ \hline 
122 & フィールドミュージアム構築における代替現実ゲーム「Ingress」の活用 & http://blog.shirai.la/wp-content/uploads/downloads/2015/04/ITSympo2014-KoseAS3.pdf & A:Dropbox (ShiraiLab)Labo2014-SemiITSympo2014 &  & 2015-04-24 15:11:00 & 711 & shirai & 0 &  \\ \hline 
123 & 無料クラウドサービスを用いたアクティブラーニング支援ツールの開発 & http://blog.shirai.la/wp-content/uploads/downloads/2015/04/ITSympo2014-Matsushita.pdf & 無料クラウドサービスを用いた
アクティブラーニング支援ツールの開発
松下 公輝・伊藤 嘉洋・白井 暁彦 &  & 2015-04-24 15:15:15 & 584 & shirai & 0 &  \\ \hline 
124 & マンガ没入型エンタテイメントシステムのためのコンテンツ製作支援ツールの開発 & http://blog.shirai.la/wp-content/uploads/downloads/2015/04/ITシンポジウム安藤0210\_2final.pdf & マンガ没入型エンタテイメントシステムのためのコンテンツ製作支援ツールの開発
安藤歩美,中澤遥,國富彦岐,小川耕作,白井暁彦 &  & 2015-04-24 15:50:16 & 540 & shirai & 0 &  \\ \hline 
125 & 博物館ネットワーク事業:特殊映像を使った展示物の開発 & http://blog.shirai.la/wp-content/uploads/downloads/2015/07/Taiyouken.pdf & 博物館ネットワーク事業:特殊映像を使った展示物の開発 &  & 2015-07-23 13:19:00 & 442 & shirai & 0 & http://sagamiharacitymuseum.jp/wp-content/uploads/2015/06/3d75dc9e1a48ce019c70d9149f1b2cc8.pdf \\ \hline 
126 & 博物館ネットワーク事業:Ingressを用いたフィールドミュージアムの開発 & http://blog.shirai.la/wp-content/uploads/downloads/2015/07/Ingress.pdf & 博物館ネットワーク事業:Ingressを用いたフィールドミュージアムの開発
小瀬 由樹, 美濃部 久美子, 白井 暁彦, 木村 知之
相模原市立博物館研究報告, ISSN 1346-3683
Vol 23集 &  & 2015-07-23 13:30:00 & 507 & shirai & 0 & http://sagamiharacitymuseum.jp/wp-content/uploads/2015/06/f9d00c0d78f044cd87f5071afcd710a3.pdf \\ \hline 
127 & 神奈川工科大学 共同研究契約書/申込書/承認願 & http://blog.shirai.la/wp-content/uploads/downloads/2015/10/jik\_kyoudou.zip & まず神奈川工科大学の共同研究制度についてはこちらのリエゾンオフィスのホームページに記載がございます。
http://www.kanagawa-it.ac.jp/~l4024/liaison/procedure.html
受託研究は目的・アウトプットがはっきりしている研究テーマ、
共同研究は、より双方の利点を生かした研究テーマとご理解ください。
(白井研究室は「共同研究」が多いです)
研究題目、研究目的及び内容は打ち合わせて考えていければと思います。
「共同研究契約書/申込書/承認願」がこのファイルです。

まずご記入いただきたいのが〔共同研究申込書〕です。

1 研究題目	
2 研究目的及び内容	
3 研究期間
4 研究実施場所
6  共同研究員
7  共同研究員派遣有無
8 経費の負担額
9 提供設備等
10 その他の条件(契約書の要否等) &  & 2015-10-09 16:15:58 & 10 & shirai & 0 &  \\ \hline 
128 & ExFieldプレスリリース20160219 & http://blog.shirai.la/wp-content/uploads/downloads/2016/02/PressRelease0219-1.zip & 神奈川工科大学が “どこから見ても正面に見える” 
広告向けディスプレイ技術を開発 ‐‐ ARや多言語表示に期待 & 2 & 2016-02-18 21:08:00 & 232 & shirai & 0 &  \\ \hline 
129 & ExField動画:2つの方向によって異なる言語の表示 & https://www.dropbox.com/s/61fhwgqk9wvkh7o/stop.mp4?dl=0 & https://www.youtube.com/watch?v=gB4iHLstMck &  & 2016-02-19 12:35:00 & 198 & shirai & 0 &  \\ \hline 
182 & Baby-PressRelease20170411 & /var/www/wordpress/wp-content/uploads/2017/04/Baby-PressRelease20170411.zip &  &  & 2017-04-11 23:17:52 & 394 & shirai & 0 &  \\ \hline 
190 & Photo09-HiRez & /var/www/wordpress/wp-content/uploads/2017/04/photo09-HiRez.png &  &  & 2017-04-11 23:17:52 & 0 & shirai & 0 &  \\ \hline 
181 & Photo09 & /var/www/wordpress/wp-content/uploads/2017/04/photo09.png &  &  & 2017-04-11 22:39:04 & 0 & shirai & 0 &  \\ \hline 
192 & 紙と鉛筆による手描きを中心としたアニメーション制作のためのスキャナ特性評価 & http://blog.shirai.la/wp-content/uploads/downloads/2017/05/IIEEJ-Fujikura.pdf & 藤倉 伊織, 柿崎 俊道, 舛本 和也, 白井 暁彦:紙と鉛筆による手描きを中心としたアニメーション制作のためのスキャナ特性評価,第 44 回画像電子学会年次大会,4 pages, 2016/6/19 &  & 2017-05-22 19:57:00 & 193 & shirai & 0 &  \\ \hline 
193 & Poster-紙と鉛筆による手描きを中心としたアニメーション制作のためのスキャナ特性評価 & http://blog.shirai.la/wp-content/uploads/downloads/2017/05/画像電子学会ポスター.pdf & Poster PDF &  & 2017-05-22 20:33:00 & 132 & shirai & 0 &  \\ \hline 
194 & Manga Generator, a future of interactive manga media : Invited Talk Paper & http://dl.acm.org/citation.cfm?id=3015156 & Akihiko SHIRAI, “Manga Generator, a future of interactive manga media : Invited Talk Paper”,
MANPU ’16 Proceedings of the 1st International Workshop on coMics ANalysis, Processing and Understanding, Article No. 13, 5 pages, 2016. [PDF] [SlideShare] &  & 2017-05-22 20:43:23 & 114 & shirai & 0 &  \\ \hline 
195 & SlideShare - Manga generator, a future of interactive manga media: invited talk paper & https://www.slideshare.net/aquihiko/manpu2016-invited-talk-manga-generator-a-future-of-interactive-manga-media & SlideShare &  & 2017-05-22 21:16:45 & 147 & shirai & 0 &  \\ \hline 
196 & SlideShare - Web-based multiplex image synthesis for digital signage & https://www.slideshare.net/aquihiko/webbased-multiplex-image-synthesis-for-digital-signage & SlideShare &  & 2017-05-22 21:27:44 & 134 & shirai & 0 &  \\ \hline 
197 & Demo - Web-based multiplex image synthesis for digital signage & https://playcanv.as/p/sipfSQO4/ & Demo &  & 2017-05-22 21:58:00 & 104 & shirai & 0 &  \\ \hline 
198 & Web-based multiplex image synthesis for digital signage & http://blog.shirai.la/wp-content/uploads/downloads/2017/05/IWAIT2017\_Web-based-multiplex-image-synthesis-for-digital-signage.pdf & Fujisawa Yoshiki, Hisataka Suzuki, Rex Hsieh and Akihiko Shirai, “Web-based multiplex image synthesis for digital signage”, Proceedings of the 20th International Workshop on Advanced Image Technology 2017 (IWAIT 2017), 3 pages. 2017. &  & 2017-05-22 22:00:00 & 222 & shirai & 0 &  \\ \hline 
 \end{longtable}
